\section*{Reviewer 5}

\paragraph{R5.1} Compared to medical equipment, our approach is less intrusive and has less disruption for a user's sleep. The use of
single wrist sensors does have its limitations, because the hand movement frequency is different on different hands. However, such a
difference would only affect the detection of the hand position and micro movements, and can be largely cancelled through calibration. As
such, the limitation has little impact on understanding the sleep quality. This is explained at Section 5.


\paragraph{R5.2} Please see M2 on how our system help users to improve their sleep quality.

\paragraph{R5.3} We use multiple ways to establish our ground truths. These include watching and labelling video footage, compared to
Fitbit, and directly user involvement. Prior work has shown that Fitbit is accurately and effectively in tracking sleep stages, hence we use it to
establish some of our ground truths. Please see also R4.1.

\paragraph{R5.4} Yes, we are not the first to use wearable devices to track sleep behaviours. However, our work shows, for the first time, how
 to track a range of events using wearable devices, and how such information can help users to improve
their sleep behaviours. We have better highlighted our novelty in Section 6. Thanks for the feedback.

\paragraph{R5.5} Agree, in theory hands can be on any place. However, we found that the arm position strongly correlates to the body
posture. This allows us to effectively map the arm position to the sleeping posture. This is clarified in Sections 2.1.1 and 3.1 in the
revised manuscript.

\paragraph{R5.6} Our current implementation considers three hand positions as they are found to be the most common and representative
positions during our user study. However, our algorithms can be extended to other positions. This is explained at Section 2.1.2 in the new
version of the manuscript.

\paragraph{R5.7} The reviewer correctly pointed out that the breathing frequency can also be used to identify sleep stages. Our work uses
the respiratory amplitude which is strongly correlated to the breathing frequency, but the breathing frequency can also be used. This is
clarified at Section 4.2.2.

\paragraph{R5.8} We have extended the discussion of our hand position detection algorithm to provide more details in Section 2.1.2. We
thank the reviewer for the feedback.

\paragraph{R5.9} A large respiratory amplitude is defined as the average respiratory amplitude (15) of the
non-rapid-eye-movement (NREM) stage, and a normal respiratory amplitude is defined as the averaged respiratory amplitude (4) of the REM
stage. The definitions are based on the prior study [48] and our observation. This is now clarified in Section 2.1.2.


\paragraph{R5.10} Large and normal respiratory events are labeled using the following method. Our current implementation only detects
respiratory events when the hand is placed on the abnormal or the chest. The label of a sleep stage, i.e., REM or NREM, is confirmed when
both Fitbit and \systemname reach a consensus. We also watch the recorded video to label some large respiratory events if these are
observable from the video footage. Our classifier for respiratory events is k-nearest neighbor model (with $k=1$). This is now
clarified in Section 2.1.2.


\paragraph{R5.11} Our description for body rollover counts is wrong, sorry. We have fixed this error in the revised version. Please see
also R1.10.


\paragraph{R5.12} In Equation 7, \emph{x} and \emph{w} are the signal and impulse response respectively. We have clarified these in
Section 2.2.

\paragraph{R5.13} Compared to existing algorithms, our methods (equations 9-12) require less training data to build and hence incur less
data collection overhead. This is now clarified in Section 2.2.

\paragraph{R5.14} The parameters of Equations 11 and 12 were determined from our training data. This is clarified in Section 2.2. See also R1.4.

\paragraph{R5.15} Our illumination threshold was also chosen based on our training data -- we have visited the bedrooms of our 10 volunteers
to take the measurements using a light meter. We stress that this threshold can be adjusted without affecting the generalization ability of
our method. This is clarified in Section 2.3.

\paragraph{R5.16} There is no prior work on addressing the unstable light readings from smartwatches. The closest one is SleepHunter, a
smartphone-based approach. However, SleepHunter relies on the proximity sensor which is typically unavailable on smartwatches. This is
explained in Section 2.3.

\paragraph{R5.17} We have given the features used for sleep stage and quality measurement in Section 2.4 in the revised manuscript.
Specificially, we use four features, \#body rollers, \#micro-body movements, respiratory amplitudes, and \#acoustic events.

\paragraph{R5.18} The time window length is set to 15 minutes when no event is detected. However, when an event (e.g., the change of the
lighting condition) occurs, we perform immediately perform a sleep-stage detection. This is now clarified at Section 4.2.1.

\paragraph{R5.19} We take a sensor reading every 30ms, which is found to be sufficient for our work. This is clarified in Section 3.1.

\paragraph{R5.20} The 15 volunteers recruited in our evaluation are indeed different from those participated in our training data collection. See also R1.4.

\paragraph{R5.21} There are studies show that Fitbit is reasonably accurate in estimating REM and detecting light sleeps. This is explained in Section 3.3.
Please see also R4.1.

\paragraph{R5.22} Yes, we did get slightly different measurements from different wrists. However, the difference has negligible impact on
understanding the sleep quality. As Sleep-Hunder and Sleep-as-Android use the sensor data from a smartphone putting on the bed, they do not
suffer from the issue of different wrist data. These are clarified in Section 5. See also R5.1.


\paragraph{R5.23} We use a window length of 15 minutes to annotate events. See also R5.18.

\paragraph{R5.24} The ``cross-validation" for sleeping postures works by training our classifier on one user and then testing it on the
remaining 14 users. We then repeat this process until all the users have been tested at least once. We have now explained this in Section
4.1.1.


\paragraph{R5.25} For SleepMonitor, we compare to the results reported by the authors. This is clarified in Section 4.1.1.

\paragraph{R5.26} For sleep position detection, we show the accuracy of sleep position detection on a per user basis (Fig.19) and the
averaged results across our 15 participants (Table 2). This is now clarified in the texts.


\paragraph{R5.27} To accurately detect cough patterns would require having an extensive set of data to capture the common patterns. We have provided a discussion in Section 4.1.5 for this issue.

\paragraph{R5.28} Yes, the values in Table 9 are the averaged measurement numbers over a period of 14 days. This is now clarified in the revised manuscript.

\paragraph{R5.29} The reviewer is right that it may be difficult to enforce the change that \systemname suggests, but we hope our system can help users to aware of the causes of poor sleeps. We have clarified this in Section 4.2.4.

\paragraph{R5.20} We have corrected the minor table reference issue pointed out the reviewer.
