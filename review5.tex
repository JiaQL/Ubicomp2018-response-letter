\section*{Reviewer 5}

\paragraph{R5.1} Compared to medical equipment, our approach is less intrusive and has less disruption for the user’s sleep. The use of
single wrist sensors does have its limitations, primarily due to the noise in the data. As such, we have designed a set of algorithms to
extract useful information from the noisy sensor data. New evidences are provided in Section 4.2.4 to show the usefulness of our approach.
We have also extended \FIXME{Section xx} to discuss the limitation of our work. Thanks for the comments.

\paragraph{R5.2} Please see M2 on how our system help users to improve their sleep quality.

\paragraph{R5.2} Please see M2 on how our system help users to improve their sleep quality.

\paragraph{R5.3} We use multiple ways to establish our ground truths. These include watching and labelling video footage, compared to
Fitbit, and directly user involvement. Prior work has shown that Fitbit is accurately in tracking sleep stages, hence we use it to
establish some of our ground truths. Please see also R4.1.

\paragraph{R5.4} Yes, we are not the first to use wearable devices to track sleep behaviours. However, our work shows that how
 to track a range of events that are not supported by prior systems, and how such information can help users to improve
their sleep behaviours. We have better highlighted our novelty in \FIXME{Section xx}.
