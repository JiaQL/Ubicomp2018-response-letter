\section*{Reviewer 5}

\paragraph{R5.1} Compared to medical equipment, our approach is less intrusive and has less disruption for the user's sleep. The use of
single wrist sensors does have its limitations, because the hand movement frequency is different on different hands. However, such a
difference would only affect the detection of the hand position and micro movements, and can be largely cancelled through calibration. Asu
such, the limitation has little impact on understanding the sleep quality. This is explained at \FIXME{Section xx}.


\paragraph{R5.2} Please see M2 on how our system help users to improve their sleep quality.

\paragraph{R5.3} We use multiple ways to establish our ground truths. These include watching and labelling video footage, compared to
Fitbit, and directly user involvement. Prior work has shown that Fitbit is accurately in tracking sleep stages, hence we use it to
establish some of our ground truths. Please see also R4.1.

\paragraph{R5.4} Yes, we are not the first to use wearable devices to track sleep behaviours. However, our work shows how
 to track a range of events that are not supported by prior systems, and how such information can help users to improve
their sleep behaviours. We have better highlighted our novelty in \FIXME{Section xx}.

\paragraph{R5.5}. Yes, in theory hands can be on any place. However, we found that the arm position strongly correlates to the body
posture. This allows us to effectively map the arm position to the sleeping posture. This is clarified in \FIXME{Section xx}.

\paragraph{R5.6} Our current implementation considers three hand positions as they are found to be the most common and representative
positions during our user study. However, our algorithms can be extended to other positions. This is explained at \FIXME{Section xx}.

\paragraph{R5.7} The reviewer correctly pointed out that the breathing frequency can also be used to identify sleep stages. Our work uses
the respiratory amplitude which is strongly correlated to the breathing frequency. This is clarified at \FIXME{Section xx}.

\paragraph{R5.8} We have extended the discussion of our hand position detection algorithm to provide more details in \FIXME{Section xx}. We
thank the reviewer for the feedback.

\paragraph{R5.9} A large respiratory amplitude is defined as \FIXME{less than three words please!} and a normal respiratory amplitude is defined as
\FIXME{XX}. These values are chosen based on the prior study \FIXME{give citation on the revised manuscript please.}. This is clarified in
\FIXME{Section xxx}.

\paragraph{R5.10} \FIXME{Briefly describe how did we label the data}. We use \FIXME{SVM} as our classifier. This information is now given in \FIXME{Section xx}.

\paragraph{R5.11} Please see R1.10 regarding the clerical error for body rollover counts. This error is fixed in the revised version.

\paragraph{R5.12} x and w are the signal and impulse response respectively in Equation 7. We have clarified this in the revised manuscript.

\paragraph{R5.13} Compared to existing algorithms, our methods (equations 9-12) require less training data to build and hence incurs less data collection overhead. This is now clarified in \FIXME{Section xx}.

\paragraph{R5.14} The parameters of equation 11 and 12 were determined from our training data. This is clarified in \FIXME{Section xx}. See also R1.4.

\paragraph{R5.15} Our illumination threshold was also chosen based on our training data -- we have visited the bedrooms of our 10 volunteers
to take measurements. This threshold can be adjusted without affecting the generalization ability of our method. This is clarified in
\FIXME{XX}.

\paragraph{R5.16} There is no prior work on addressing the unstable light readings from smartwatches. The closest one is SleepHunter, a
smartphone-based approach. However, SleepHunter relies on the proximity sensor which is typically unavailable on smartwatches. This is
explained in \FIXME{Section xx}.

\paragraph{R5.17} We have given the features used for sleep stage and quality measurement in \FIXME{Section xx} in the revised manuscript.
Specificially, we use four features, \#body rollers, \#micro-body movements, respiratory amplitudes, and \#acoustic events.

\paragraph{R5.18} The time window length is set to 15 minutes when no event is detected. However, when an event (e.g., the change of the
lighting condition) occurs, we perform immediately perform a sleep-stage detection. This is now clarified at \FIXME{Section xx}.

\paragraph{R5.19} We take a sensor reading every 30ms, which is found to be sufficient for our work. This is clarified in \FIXME{Section xx}.

\paragraph{R5.20} The 15 volunteers recruited in our evaluation are indeed different from those participated in our training data collection. See also R1.4.

\paragraph{R5.21} There are studies show that Fitbit is quite accurate in estimating rapid-eye-movement (REM) and detecting light sleeps. Please see also R4.1.

\paragraph{R5.22} Yes, we did get slightly different measurements from different wrists. However, the difference has negligible impact on
understanding the sleep quality. As Sleep-Hunder and Sleep-as-Android use the sensor data from a smartphone putting on the bed, they do not
suffer from the issue of different wrist data. These are clarified in \FIXME{Section xx}. See also R5.1.


\paragraph{R5.23} We use a window length of 15 minutes to annotate events. See also R5.18.

\paragraph{R5.24} Our ``cross-validation” works by training our classifier on one user and then testing it on the remaining 14 users. We then repeat this process until all the users have been tested at least once. We have now explained this in \FIXME{Section xxx}.


\paragraph{R5.25} For SleepMonitor, we compare to the results reported by the authors. This is clarified in \FIXME{Section xx}.

\paragraph{R5.26} \FIXME{Why did the authors choose only one participant show the performance of body posture detection??? – Why not show the performance of all users????}

\paragraph{R5.27} To accurately detect cough patterns would require having an extensive set of data to capture the common patterns. We have provided a discussion in \FIXME{Section xx} for this issue.

\paragraph{R5.28} Yes, the values in Table 9 are the averaged measurement numbers over a period of 14 days. This is now clarified in the revised manuscript.

\paragraph{R5.29} \systemname can suggest a user to avoid some hand positions and the reviewer is right that it may be difficult to enforce such a change. We have clarified this in \FIXME{Section xx}.

\paragraph{R5.20} We have corrected the minor reference issue pointed out the reviewer.
