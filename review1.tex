\section*{Reviewer 1}

\paragraph{R1.1} We have extended the introduction to discuss the advantages and disadvantages of PSG and why it is not applicable to our
problem. Thanks for the comments.

\paragraph{R1.2} Embarrassed not to have introduced and provided sufficient details on our pilot study. This information is now given in Section 3.

\paragraph{R1.3} We take 12-20 samples every minute to detect respiratory events. This is based on the observation that an adult typically
breathes around 15 times per minute. We found that our sampling range is effective for our problem. This is now clarified in \FIXME{Section
xx}.

\paragraph{R1.4} The 10 volunteers involved in our pilot study for determining the algorithm parameters are different from the 15 users recruited in our
evaluation. The data were collected while they were sleeping. This is clarified in \FIXME{Section xx}.

\paragraph{R1.5} Our HMM model is trained using the data collected from the 10 volunteers took part in our pilot study. The 10 volunteers
are different from those involved in our evaluation. This is now clarified in \FIXME{Section xx}.

\paragraph{R1.6} To evaluate our approach, we have randomly picked at least 3 sets of data from each of our 15 volunteers recruited for our
evaluation. We do not evaluate our approach on the full dataset due to the time constraint. However, we are able to demonstrate the
usefulness of our approach on a relatively small but representative set of data. This is explained in \FIXME{Section xx} of the revised
manuscript. Please see also response to R4.2. \FIXME{ZW: I think we need a better answer than just saying we couldn't be asked to do
experiments on all the 210 datasets. For examples, are there any (nearly) identical datasets?}

\paragraph{R1.7} ``invasive" is indeed a poorly chosen word for describing Fibit. We have reworded.

\paragraph{R1.8} The signals shown in Fig.5 are from a single subject. This is clarified in the revised version.
