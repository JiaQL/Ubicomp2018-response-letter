\section*{Reviewer 4}

\paragraph{R4.1} We agree with the reviewer that it would be interesting to compare our work against a PSG if possible. However, doing so is infeasible because of the focus of the work and practical reasons. Our work focuses on tracking a user's physical activities (like hand positions and body rollover) and the sleeping environment (like lighting and acoustic events), and we need to achieve these with as little disruption to the user's normal sleep as possible. A PSG would only track the user's sleep stages. %It is not only expensive to buy, but will also cause significant disruptions to a user's normal sleep. These make it difficult to perform a large-scale experiment across a long-period of time using PSGs. Instead, we choose to compare our work with Fitbit as its accuracy is shown to be comparable to a PSG. \textcolor{blue}{The detailed reason why we choose Fitbit can refer to R2.1.} 
\textcolor{blue}{ And we chose Fitbit as the groundtruth of the assessment system, instead of PSG, for the following reasons: Firstly,  Fitbit is a commercial state-of-the-art and a popular sleep monitoring wristband. It has less interference with the subject's normal sleep process, making easier for subjects to relax and maintain normal sleep. But PSG  must wear a large number of instruments on the subject's body, which may have a certain impact on their psychological and normal sleep processes, so as not to reflect the actual sleep conditions. Therefore, even if the data measured by the PSG is very accurate, the reliability of data and results is also reduced due to invasiveness to sleep. Secondly, Fitbit is inexpensive and easy to deploy, so it is very convenient to apply to our large number of experimental scenarios. And it shows better performance in this kind of products based on actigraphy. Thirdly, there were some previous work~\cite{fitbit01,fitbit02,fitbit03} demonstrating Fitbit to have a high sensitivity for sleep, good association of movement measurements and to be comparable to PSG in adults. Although they also pointed out that Fitbit has some significant limitations, it is still very satisfactory in estimating REM and light sleep stage.}
Moreover, we also show that our approach can track more events and lead to a better understand of the sleep quality than Fitbit. We have clarified and provided a discussion on this in Section 6. Please see also R4.3. We thank the reviewer for the feedback.

\paragraph{R4.2} We have performed further analysis on our user study per the review feedback. Please refer to M2. We have also reworded our
claims. Thank you.


\paragraph{R4.3} We have provided a discussion on the advantages and disadvantages for PSG in Section 1. See also R1.1. Thanks for the feedback.

\paragraph{R4.4} Please see our response to R1.2 regarding our pilot study.

\paragraph{R4.4} Please refer to R1.3 on how the thresholds of respiration detection are chosen.

\paragraph{R4.5} The volunteers recruited for determining our algorithm parameters are different from those involved in
 evaluation. The details are provided in Section 2.1 of the revised manuscript. See also R1.4.

 \paragraph{R4.6} Please see our response to R1.5 on how the HMM model is built.
