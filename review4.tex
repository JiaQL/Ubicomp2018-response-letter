\section*{Reviewer 4}

\paragraph{R4.1} We agree with the reviewer that it would be interesting to compare our work against a PSG if possible. However, doing so
is infeasible because of the focus of the work and practical reasons. Our work primarily focuses on tracking a user's physical activities
(like hand positions and body rollover) and the sleeping environment (like lighting and acoustic events), and we need to achieve these with
as little disruption to the user's normal sleep as possible. A PSG only tracks a user's sleep stages, but will cause significant
disruptions to a user's normal sleep routine (as it requires the user to wear some instruments). The disruption makes PSG unsuitable for a
large experiment across a period of time. As a compromise, we have chosen to compared with Fitbit as it is less intrusive, and proven to be
accurate in estimating the rapid-eye-movement (REM) stage, light sleep stages, and gathering movement
measurements [15][27][52]. We have clarified and provided a discussion on this in Section 1 and Section 3.3. Please see also R4.3. We
thank the reviewer for the feedback.

\paragraph{R4.2} We have performed further analysis on our user study per the review feedback. Please refer to M2. We have also reworded our
claims. Thank you. \vspace{-2mm}

\paragraph{R4.3} We have provided a discussion on the advantages and disadvantages for PSG in Section 1. See also R1.1. Thanks for the feedback.
\vspace{-2mm}
\paragraph{R4.4} Please see our response to R1.2 regarding our pilot study.
\vspace{-2mm}
\paragraph{R4.4} The thresholds of respiration detection are determined by analyzing our training data. Please
see also R1.3.
\vspace{-2mm}

\paragraph{R4.5} The volunteers recruited for determining our algorithm parameters are different from those involved in the
 evaluation. The details are now provided in Section 2.1.2 of the revised manuscript. See also R1.4.

 \paragraph{R4.6} Please see our response to R1.5 on how the HMM model is built.
